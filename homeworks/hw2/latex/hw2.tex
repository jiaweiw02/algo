\documentclass{article}
\usepackage[margin=1.25in]{geometry}
\usepackage{amsmath, amssymb, setspace, enumerate, enumitem}
\usepackage{setspace}
\usepackage{graphicx}
\doublespacing

\begin{document}
    \begin{enumerate}[label=\textbf{Q1}]
        \item Give an algorithm (pseudo code, with explanation) to compute $2^{2^n}$ in linear time, assuming multiplication of arbitrary size integers takes unit time. What is the bit-complexity if multiplications do not take unit time, but are a function of the bit-length. \\
        \textbf{Solution:}\\
        Algorithm to compute $2^{2^n}$ in linear time.
        \begin{verbatim}
mult(n):
    Inputs: the value n
    Outputs: the product

    x = 2 # base case

    for i in n:
        x = x squared
    
    return x
        \end{verbatim}
        Assuming that arbitrary size integers multiplication take unit time,
        we have $arr[i-1]$ multiplied by $arr[i-1]$. If they are O(1) and the
        loop runs n times, then we have $O(n)$, which is linear time.

    \end{enumerate}
\end{document}