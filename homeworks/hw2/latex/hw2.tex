\documentclass{article}
\usepackage[margin=1.25in]{geometry}
\usepackage{amsmath, amssymb, setspace, enumerate, enumitem}
\usepackage{setspace}
\usepackage{graphicx}
\doublespacing

\begin{document}
    \begin{enumerate}[label=\textbf{Q1}]
        \item Give an algorithm (pseudo code, with explanation) to compute $2^{2^n}$ in linear time, assuming multiplication of arbitrary size integers takes unit time. What is the bit-complexity if multiplications do not take unit time, but are a function of the bit-length. \\
        \textbf{Solution:}\\
        Algorithm to compute $2^{2^n}$ in linear time.
        \begin{verbatim}
mult(n):
    a = 1 shifted to the left n bits
    return 1 shifted to the left a bits
        \end{verbatim}
        The bit complexity if multiplication does not take unit time would be
        $2^{2n}$.
    \end{enumerate}

    \begin{enumerate}[label=\textbf{Q2}]
        \item Consider the problem of computing $N! = 1 \cdot 2 \cdot 3 \cdot \cdot \cdot N$
        \begin{enumerate}[label=(\alph*)]
            \item If $N$ is an n-bit number, how many bits long is $N!$ in $O()$ notation (give the tightest bound)?\\
            \textbf{Solution:}\\
            Since $N$ is an n-bit number. We assume that $N \times N$ will be $\Theta(N^2)$ time
            complexity. However, since it is a factorial, the value multiplied will begin to get smaller.
            Since there is a decrease in the value of $N$ being multiplied, then the total running
            time for $N!$ will be $\Theta(nlogn)$
            \item Give an algorithm to compute $N!$ and analyze its running time.\\
            \textbf{Solution:}
            \begin{verbatim}
nfactorial(n):
    int result = 0;
    for (int i = 2; i < n; i++){
        result *= i;
    }
    return result;
            \end{verbatim}
            The for loop will run $n$ times, multiplication $n \times n$ would be $n^2$ runtime,
            but since our multiplication is by an increasing value of $n$, the first numbers leading
            up to $n$ are negligeable until reaching to n. Instead of the multiplication being $n$, 
            it can be considered as $log n$, taking a total of $nlogn$ running time to complete.
        \end{enumerate}
    \end{enumerate}
\end{document}