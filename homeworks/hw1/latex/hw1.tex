\documentclass{article}
\usepackage[margin=1.25in]{geometry}
\usepackage{amsmath, amssymb, setspace, enumerate, enumitem}
\usepackage{setspace}
\usepackage{graphicx}
\onehalfspacing

\begin{document}
    \begin{enumerate}[label=\textbf{Q1}]
        \item (20 points; 2 per part) DPV Problem 0.1 (parts a through j): In each of the following situations, indcate whether $f = O(g)$, or $f = \Omega(g)$, or both (in each case $f = \theta(g)$).\\[0.25in]
        \begin{tabular}{l l l l}
            &$f(n)$ & $g(n)$ & \textbf{answer}\\
            \textbf{(a)} & $n - 100$ & $n - 200$ & $f = \theta(g)$\\
            \textbf{(b)} & $n^{\frac{1}{2}}$ & $n^{\frac{2}{3}}$ & $f = O(g)$\\
            \textbf{(c)} & $100n + log\ n$ & $n + (log\ n)^2$ & $f = \theta(g)$\\
            \textbf{(d)} & $n\ log\ n$ & $10n\ log\ 10n$ & $f = \theta(g)$\\
            \textbf{(e)} & $log\ 2n$ & $log\ 3n$ & $f = \theta(g)$\\
            \textbf{(f)} & $10\ log\ n$ & $log(n^2)$ & $f = \theta(g)$\\
            \textbf{(g)} & $n^{1.01}$ & $nlog^2\ n$ & $f = \Omega(g)$\\
            \textbf{(h)} & $\frac{n^2}{log\ n}$ & $n(log\ n)^2$ & $f = \Omega(g)$\\
            \textbf{(i)} & $n^{0.1}$ & $(log\ n)^{10}$ & $f = \Omega(g)$\\
            \textbf{(j)} & $(log\ n)^{log\ n}$ & $\frac{n}{log\ n}$ & $f = \Omega(g)$\\
        \end{tabular}
    \end{enumerate}

    \begin{enumerate}[label=\textbf{Q2}]
        \item (10 points) Consider the following pseudo-code which takes the integer $n \geq 0n \geq 0$ as input:
        \begin{verbatim}
            Function bar(n)
                Print '*';
                if n == 0 then
                    Return;
                end
                for i = 0 to n - 1 do
                    bar(i);
                end
        \end{verbatim}
        Let $T(n)$ be the number of times the above function prints a star ($*$) when called with input $n \geq 0$.
        What is $T(n)$ exactly, in terms of only $n$ (and not values like $T(n - 1)$ or $T(n - 2))$? Prove your
        statement\\[0.25in]
        \textbf{Solution} $= \mathbf{2^n}$\\
        \begin{tabular}{l l l}
            $n$ & calls & $T(n)$\\
            $0$ & print & $T(0) = 1$\\
            $1$ & print, $T(0)$ & $T(1) = 2$\\
            $2$ & print, $T(0)$, $T(1)$ & $T(2) = 4$\\
            $3$ & print, $T(0)$, $T(1)$, $T(2)$ & $T(3) = 8$\\
        \end{tabular}

        We deduct that $T(n) = 2^n$\\
        We can prove the claim by induction. Our recurrence formula given by the function is $T(n) = 2T(n-1)$ for 
        $T(0) = 1$.\\
        Base Case: $n = 0$
        \begin{align*}
            T(0) &= 1\\
            1 &= 2^0\\
            1 &= 1
        \end{align*}
        Induction step: prove $2^n = 2T(n-1)$ for $n = k+1$
        \begin{align*}
            \boxed{2^{k+1}} &= 2T(k+1-1)
        \end{align*}
        \centerline{Work on RHS}
        \begin{align*}
            2T(k) &= 2 \times 2^k\\
            2 \times 2^k &= \boxed{2^{k+1}}
        \end{align*}
        Thus proving that our recurrence formula generates the given solution $2^n$.

    \end{enumerate}

    \begin{enumerate}[label=\textbf{Q3}]
        \item (30 points) Let $f(n)$ and $g(n)$ be asymptotically nonnegative functions. Using the basic
        definition of $\theta$-notation, prove that $max(f(n),g(n))= \theta(f(n)+g(n))$\\[0.25in]
        Basic notation of
    \end{enumerate}

    \begin{enumerate}[label=\textbf{Q4}]
        \item (10 points; 5 per part)
        \begin{enumerate}[label=(\alph*)]
            \item is $2^{2n} = O(2^n)$?\\
            No
            \item why?\\
            $n^a$ dominates $n^b$ if $a > b$. In this instance, no matter the value for $n$, $2n$ will always
            dominate $n$ (unless negative, but that doesn't apply here). In addition, $2^{2n}$ can be rewritten 
            as $(2^2)^n$, therefore it is in $O(4^n)$.
        \end{enumerate}
    \end{enumerate}
\end{document}